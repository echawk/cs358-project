\documentclass{article}

\usepackage{hyperref}

\title{Progress Report 1\\Satellite Observation Scheduling Simulator}
\author{Erik, Ethan, Nick}

\begin{document}
\maketitle

\section{Objectives Completed}

\begin{itemize}
	\item System Overview Document
	\item Meeting with Professor White
	\item inital email to communicate with SatNOGs devs.
	\item decided to use flask as web framework (may switch to django if satnogs team has resources for us)
\end{itemize}

\section{Current Objectives}

\begin{itemize}
  \item Read in data from SatNOGs website
  \item Client-Server Architecture
    \begin{itemize}
      \item methods needed to interface client server
    \end{itemize}
  \item Bidding system ideas
  \item Wait for Prof. White's response to the requirements document
  \item Wait for a response from the SatNOGs team
\end{itemize}

\section{Meeting w/ Professor White}

\subsection{Base station preference features}

\begin{itemize}
\item Be able to identify an obstacle specific to base station
  \begin{itemize}
    \item segments that are boxed out and realized as pointless (part of the preferences of the station)
  \end{itemize}
\item Uses a pass predictor because the network does not supply all the needed info for the station client
  \item Pass predictor uses time, freq range and area for preference
\end{itemize}

\subsection{Pass Prediction features}

\begin{itemize}
\item Pass prediction must be above horizon
\item Make sure that positioning is good
\item DON'T bother listening until it is high enough \& therefore closer
\item Can't get pass predictions via api, BUT we can use API root to ``poke at it direct''
\end{itemize}

\subsection{SatNOGs Communication}

\begin{itemize}
\item Karel - username of person we should initially reach out to
\item Reach out to SatNOGs devs - also could use matrix channels for SatNOGs \& librespace
\end{itemize}


\subsection{Notes for NEW scheduler}

\begin{itemize}
\item If it fits in the base station calendar, do it!
\item Be able to plug in any method that does fancier and fancier stuff \textbf{VERY IMPORTANT}
\item More interested in the engine that allows us to plug in a client and behave a certain way
\item Start with the auto scheduler, one owner, one groundstation centric
\item Scheduler is only a part of the simulator
\item Atuo Scheduler runs its own predictor, but you can use api to get all stations, all satellites and pass predictions for all satellites individually giving a list of all possible passes
  \begin{itemize}
  \item We will be delivered a database with all of that information from Prof. White
    \item No need to run in real time
  \end{itemize}
\item We want to be able to auto schedle a satelite OVER MANY GROUNDSTATIONS
\end{itemize}

\subsection{APIs}

In the simulator we get access to all ground stations via api, transmitters via API (satellites can have multiple transmitters)

\href{db.satnogs.org/api}{db.satnogs.org/api}

\href{db.satnogs.org}{db.satnogs.org} - SatNOGs database


\subsection{Auction/Bidding System}

Auction capabilites would use another plugin.


\subsection{Questions Answered}

Is tehre any potential for code reuse from previous bazaar project work?

\begin{itemize}
  \item No, previous groups were not CS groups and had very bare-bones system
  \item Could MAYBE use datastructures but don't use more than that
  \item Should instead start with the SatNOGs auto scheduler
\end{itemize}

What would the best case scenario for a final user interface look like?

\begin{itemize}
  \item Have a way of describing behavior and priorities, priority list for satellite owners
  \item If we want to listen to an entire cluster we would want to organize reading all of them with N number of groundstations - Coverage of many different satellites over a certain area (hear something from all the satellites)
  \item Doest really have a UI, allows someone to game out a set of parties in teh bazaar and run the simulation
  \item For the same time frame and same satellites it is deterministic
\end{itemize}

\end{document}
