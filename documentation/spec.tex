\documentclass{article}

\author{Ethan Hawk, Nick Koeppen, Erik Petersen}
\title{SatNOGs Bazaar Scheduler}

\begin{document}

\maketitle
\tableofcontents

\section{General Information}

SatNOGS is an open-source project which contains a database consisting of many
satellites and observers, allowing ground station owners to manually schedule
communications between it and overhead satellites, enabling the satellite owner
to acquire data from the satellite. As it stands, the SatNOGS schedule relies
entirely on observers manually scheduling satellites that are predicted to pass
over the ground station. The project was tasked with providing the framework
towards a more automated solution for the SatNOGS community, working alongside
both the team at Valparaiso University and the SatNOGS developers. The proposed
solution is a framework that will allow the individual owners of the satellites
and the ground stations to make compromises allowing greater flexibility in the
priorities of scheduling a job. This turns the current scheduling system into a
more versatile and automated auction-style system defined by the user that
removes the need for end-users to manually schedule individual overhead
satellite communications.

\subsection{Quality Objectives}

To evaluate quality, the documentation should provide any future implementers
and SatNOGS community members enough information, as well as coherent enough
information to pick up where we left off. Particularly with regards to the
implementation of this software system.

\subsection{Use Case Description}

\subsection{Definitions}

\begin{itemize}
  \item \textbf{Job} - A satellite pass over a particular groundstation
\end{itemize}

\subsubsection{Data Layout}


\subsection{Scheduler Description}

For a high level overview of the Scheduler, at it's most basic level, one can
think of it as being a function that takes in a table of all valid potential
satellite jobs and returns a table of ``scheduled'' jobs.

\subsubsection{Scheduler Architecture}

The entire scheduler system consists of 3 parts which work in tandem to produce
a table of scheduled jobs. Which consists of the following:

\begin{itemize}
\item Job Distributor
\item Ground Station Job Evaluator
\item Job Aggregator
\end{itemize}

\subsubsection{Job Distributor}

The job distributor is in charge of sending valid \textit{Jobs} to their
respective \textit{groundstations} to be evaluated.

\noindent
Inputs:

\begin{itemize}
  \item A table of all valid \textit{Jobs}
  \item A list of all known \textit{groundstations}
\end{itemize}


\subsubsection{Ground Station Job Evaluator}



\subsection{Test Plan}



\subsection{Future Plans}

\end{document}
