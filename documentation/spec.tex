\documentclass{article}

\author{Ethan Hawk, Nick Koeppen, Erik Petersen}
\title{SatNOGs Scheduling Bazaar}

\begin{document}

\maketitle
\tableofcontents

\section{Acknowledgements}

Our work for an automated SatNOGS scheduler would not have been possible without the vast guidance and assitance from Professor Nicholas S. Rosasco, D.Sc., of the Valparaiso University Computer Science Department, and Professor Dan White Ph.D., of the Valparaiso University Electical Engineering Department. We would like to acknoweledge and thank them for their time and contribution of knowledge.


\section{General Information}

SatNOGS is an open-source project which contains a database consisting of many
satellites and observers, allowing ground station owners to manually schedule
communications between it and overhead satellites, enabling the satellite owner
to acquire data from the satellite. As it stands, the SatNOGS schedule relies
entirely on observers manually scheduling satellites that are predicted to pass
over the ground station. The project was tasked with providing the framework
towards a more automated solution for the SatNOGS community, working alongside
both the team at Valparaiso University and the SatNOGS developers. The proposed
solution is a framework that will allow the individual owners of the satellites
and the ground stations to make compromises allowing greater flexibility in the
priorities of scheduling a job. This turns the current scheduling system into a
more versatile and automated auction-style system defined by the user that
removes the need for end-users to manually schedule individual overhead
satellite communications.

\section{Quality Objectives}

To evaluate quality, the documentation should provide any future implementers
and SatNOGS community members enough information, as well as coherent enough
information, to pick up where we left off. Particularly with regards to the
implementation of this software system.


\section{Future Plans}

\subsection{Job Distributor as Abstract Entity}

Our current implementation entails a proof-of-concept, exemplifying an automated platform for ground stations to schedule satellites via sigmoid-compiled preferences. However, when implemented, we would recommend to seperate the job scheduler role from the ground station. By having the job distributor as a more abstract entity, a bigger picture of ground station accessibility can be seen and scheduled for. For example, if a satellite cluster is hovering over Europe, which has numerous groundstations, it can be more efficiently scheduled for one-on-one communication by an abstract entity as opposed to a scheduler led by a ground station, because the seperate entity can better account for the mass of satellites and ground stations and can schedule for what relationships work best, instead of what works best for a certain ground station. Subsequently, when explaining the roles working within our scheduler in this document, we will define the system in-terms of this abstract entity.

\subsection{Currency to Allow for Decision Override}

With the current implementation that SatNOGS provides, satellites have absolutely no say in how or when they are scheduled. It is entirely up to the ground station in how the satellite is manually scheduled. In our current concept, the satellite could sacrifice the quantity of communication it gets for the quality, perhaps by being not scheduled once hourly with Software-Defined-Radio equipped Raspberry Pi's and instead being scheduled once daily with high-power antenna setups. This enables the satellite to have its own say in how and when it is scheduled, even though it is not the final voice in the matter. We would recommend retaining this model, but would encourage the implementation of a 'currency' which would enable the satellite to further influence a scheduling decision, perhaps because the satellite owner requires data from the satellite in a timely fashion.

\section{Use Case Description}

Through this scheduling program, users of the SatNOGS database could eventually (if implemented into the SatNOGS website) be able to automate the processes in which their communications are scheduled. Furthermore, both ground stations and satellites would be enabled to declare preferences, which can influence the outcome of how they are scheduled as well. This, in our rendition of the scheduler, would occur via weights towards certain biases. For example, if a satellite sees that a ground station has undesirable reliability ratings in its communications, it could sacrifice less communication on the network for higher reliability. The same could also work on behalf of ground stations judging satellites. Another potential way in which preferences could affect scheduling would be if there is a congruence between the interests of the satellite and the interests of the ground station. For example, potentially higher academia, such as Valparaiso University's own SatNOGS ground stations, wish to primarily use their communication time for satellites who are interested in research. Subsequently, satellites with more research-centric goals would be far more likely (not to be confused with 'guaranteed', however) to be scheduled than those with other intents.  

\section{Scheduler Description}

For a high level overview of the Scheduler, at it's most basic level, one can
think of it as being a function that takes in a table of all valid potential
satellite jobs and returns a table of ``scheduled'' jobs.

\subsection{Scheduler Architecture}

The entire scheduler system consists of 3 parts which work in tandem to produce
a table of scheduled jobs. Which consists of the following:

\begin{itemize}
\item Job Distributor
\item Ground Station Job Evaluator
\item Job Aggregator
\end{itemize}

\subsection{Job Distributor}

The job distributor is in charge of sending valid \textit{Jobs} to their
respective \textit{groundstations} to be evaluated.

\noindent
Inputs:

\begin{itemize}
  \item A table of all valid \textit{Jobs}
  \item A list of all known \textit{groundstations}
\end{itemize}


\subsection{Ground Station Job Evaluator}

The ground station job evaluator decides what jobs to take and what jobs to
deny. It is simply a peice of software that reads in potential jobs, and
decides what jobs to take based off of the groundstation owners preferences.

\subsection{Data Layout}

Data is input through the program via an SQL database file, and the provided 2018 example database from SatNOGS was used for testing. As it stands, output is created and organized through a CSV, which makes it very accessible and easy to parse through if needed. 



\end{document}
